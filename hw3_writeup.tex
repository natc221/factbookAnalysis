\documentclass[letterpaper]{article}
\usepackage{amsfonts}
\usepackage{enumitem}
\usepackage{mathtools}
\usepackage{fancyhdr}
\usepackage{listings}
\usepackage{color}

\definecolor{dkgreen}{rgb}{0,0.6,0}
\definecolor{gray}{rgb}{0.5,0.5,0.5}
\definecolor{mauve}{rgb}{0.58,0,0.82}

\lstset{frame=tb,
  language=Java,
  aboveskip=3mm,
  belowskip=3mm,
  showstringspaces=false,
  columns=flexible,
  basicstyle={\small\ttfamily},
  numbers=none,
  numberstyle=\tiny\color{gray},
  keywordstyle=\color{blue},
  commentstyle=\color{dkgreen},
  stringstyle=\color{mauve},
  breaklines=true,
  breakatwhitespace=true
  tabsize=3
}
\begin{document}

\pagestyle{fancy}
\lhead{NETS 150 Homework 3}
\rhead{Nathaniel Chan}

\textbf{Part 1: Theory}
\begin{enumerate}
\item A socket is the endpoint of a network connection. A connection attaches a socket that is bound to a specific port. A \texttt{ServerSocket} is implemented for a server, and it listens to a connection request from a client. A \texttt{Socket} is used for a client, and is the other end of the connection. It is also bounded to a particular port.

\item When the port numbers do not match, the connection will not be established, and it will throw a ConnectException, seen as follows:

\texttt{java.net.ConnectException: Connection refused}

Or if there is already a server that is awaiting a connection at the specified port, then it will connect to that ServerSocket, but incorrectly.

\item Only the \texttt{doSomething()} method is modified from the \texttt{Server.java} from class, seen as follows:

\begin{lstlisting}
private void doSomething() {
	while (in.hasNextLine()) {
		String message = in.nextLine();
		System.out.println("The Client says: " + message);

		if (message.contains("ADD") || message.contains("MUL")) {
			String[] split = message.split(" ");
			try {
				double x = Double.parseDouble(split[1]);
				double y = Double.parseDouble(split[2]);
				double result = 0;

				if (split[0].equals("ADD")) {
					result = x + y;
				}
				if (split[0].equals("MUL")) {
					result = x * y;
				}
				out.println("Answer: " + result);
			}
			catch (NumberFormatException e) {
				out.println("Error. Wrong format.");
			}
		}
	}
}
\end{lstlisting}

\end{enumerate}

\newpage
\textbf{Part 2: Programming}
\begin{enumerate}
\item Argentina,
Chile,
Colombia,
Ecuador,
Peru

\item United Kingdom

\item Angola
Australia
Botswana
Burundi
Christmas Island
Cocos (Keeling) Islands
Comoros
Congo, Democratic Republic of the
Congo, Republic of the
Fiji
Indonesia
Kenya
Kiribati
Lesotho
Madagascar
Malawi
Mauritius
Mozambique
Namibia
New Caledonia
New Zealand
Norfolk Island
Papua New Guinea
Rwanda
Seychelles
Solomon Islands
South Africa
Swaziland
Tanzania
Timor-Leste
Tuvalu
Vanuatu
Zambia
Zimbabwe



\item Hong Kong,
India,
Kazakhstan,
Malaysia,
Maldives,
Nepal,
Pakistan,
Thailand

\item Iceland : 50684 kWh

Norway : 23461 kWh

Kuwait : 16105 kWh

Finland : 14747 kWh

Canada : 14603 kWh

Sweden : 13937 kWh

United Arab Emirates : 13540 kWh

Qatar : 12915 kWh

United States : 12280 kWh

Luxembourg : 11601 kWh

\item \textbf{More than 80\%:}

Algeria: Sunni Muslim (state religion) 99\%

Anguilla: Protestant 83.1\% (Anglican 29\%

Argentina: nominally Roman Catholic 92\% (less than 20\% practicing)

Armenia: Armenian Apostolic 94.7\%

Aruba: Roman Catholic 80.8\%

Azerbaijan: Muslim 93.4\%

Bahrain: Muslim (Shia and Sunni) 81.2\%

Bangladesh: Muslim 89.5\%

Bolivia: Roman Catholic 95\%

British Virgin Islands: Protestant 84\%

Burma: Buddhist 89\%

Cambodia: Buddhist (official) 96.4\%

Colombia: Roman Catholic 90\%

Comoros: Sunni Muslim 98\%

Croatia: Roman Catholic 87.8\%

Cuba: nominally Roman Catholic 85\%

Curacao: Roman Catholic 80.1\%

Denmark: Evangelical Lutheran (official) 95\%

Djibouti: Muslim 94\%

Dominican Republic: Roman Catholic 95\%

Ecuador: Roman Catholic 95\%

Egypt: Muslim (mostly Sunni) 90\%

Faroe Islands: Evangelical Lutheran 83.8\%

Finland: Lutheran Church of Finland 82.5\%

France: Roman Catholic 83\%-88\%

Gambia, The: Muslim 90\%

Gaza Strip: Muslim (predominantly Sunni) 99.3\%

Georgia: Orthodox Christian (official) 83.9\%

Greece: Greek Orthodox (official) 98\%

Guam: Roman Catholic 85\%

Guinea: Muslim 85\%

Honduras: Roman Catholic 97\%

Hong Kong: eclectic mixture of local religions 90\%

Iceland: Lutheran Church of Iceland (official) 80.7\%

India: Hindu 80.5\%

Indonesia: Muslim 86.1\%

Iran: Muslim (official) 98\%

Iraq: Muslim (official) 97\%

Ireland: Roman Catholic 87.4\%

Japan: Shintoism 83.9\%

Jordan: Sunni Muslim 92\% (official)

Kuwait: Muslim (official) 85\%

Liberia: Christian 85.6\%

Libya: Sunni Muslim (official) 97\%

Luxembourg: Roman Catholic 87\%

Malawi: Christian 82.7\%

Mali: Muslim 94.8\%

Malta: Roman Catholic (official) 98\%

Mauritania: Muslim (official) 100\%

Mexico: Roman Catholic 82.7\%

Moldova: Eastern Orthodox 98\%

Monaco: Roman Catholic 90\% (official)

Morocco: Muslim 99\% (official)

Namibia: Christian 80\% to 90\% (at least 50% Lutheran)

Nepal: Hindu 80.6\%

Pakistan: Muslim (official) 96.4\% (Sunni 85-90\%)

Panama: Roman Catholic 85\%

Paraguay: Roman Catholic 89.6\%

Peru: Roman Catholic 81.3\%

Poland: Roman Catholic 89.8\% [about 75\% practicing]

Portugal: Roman Catholic 84.5\%

Puerto Rico: Roman Catholic 85%

Saint Pierre and Miquelon: Roman Catholic 99\%

Saint Vincent and the Grenadines: Protestant 75\% (Anglican 47\%)

Saudi Arabia: Muslim (official) 100\%

Senegal: Muslim 94\%

Serbia: Serbian Orthodox 85\%

Seychelles: Roman Catholic 82.3\%

Spain: Roman Catholic 94%

Sweden: Lutheran 87\%

Taiwan: mixture of Buddhist and Taoist 93\%

Tajikistan: Sunni Muslim 85\%

Thailand: Buddhist (official) 94.6\%

Timor-Leste: Roman Catholic 98\%

Turkey: Muslim 99.8\% (mostly Sunni)

Turkmenistan: Muslim 89\%

Tuvalu: Protestant 98.4\% (Church of Tuvalu (Congregationalist) 97\%

Uzbekistan: Muslim 88\% (mostly Sunni)

Venezuela: nominally Roman Catholic 96\%

Virgin Islands: Protestant 59\%

Wallis and Futuna: Roman Catholic 99\% \\

\newpage
\textbf{Less than 50\%:}

Angola: indigenous beliefs 47%

Belize: Roman Catholic 39.3%

Benin: Catholic 27.1%

Bosnia and Herzegovina: Muslim 40%

Cameroon: indigenous beliefs 40%

Canada: Roman Catholic 42.6%

Central African Republic: indigenous beliefs 35%

Christmas Island: Buddhist 36%

Cote d'Ivoire: Muslim 38.6%

Czech Republic: Roman Catholic 10.3%

Estonia: Evangelical Lutheran 13.6%

Ethiopia: Ethiopian Orthodox 43.5%

Germany: Protestant 34%

Jamaica: Protestant 62.5% (Seventh-Day Adventist 10.8%

Korea, South: Christian 31.6% (Protestant 24%

Latvia: Lutheran 19.6%

Mauritius: Hindu 48%

Mozambique: Catholic 28.4%

Netherlands: Roman Catholic 30%

Palau: Roman Catholic 41.6%

Papua New Guinea: Roman Catholic 27%

Russia: Russian Orthodox 15-20%

Singapore: Buddhist 42.5%

Sint Maarten: Roman Catholic 39%

Suriname: Hindu 27.4%

Swaziland: Zionist 40% (a blend of Christianity and indigenous ancestral worship)

Switzerland: Roman Catholic 41.8%

Togo: Christian 29%

Tonga: Christian (Free Wesleyan Church claims over 30

Trinidad and Tobago: Roman Catholic 26%

Uganda: Roman Catholic 41.9%

Uruguay: Roman Catholic 47.1%

Vietnam: Buddhist 9.3%

\newpage
\item Holy See (Vatican City),
Lesotho,
San Marino

\item Top 10 countries with the highest military expenditure per capita:

Qatar : \$9351

Oman : \$3308

United Arab Emirates : \$3224

Saudi Arabia : \$3113

Israel : \$2427

United States : \$2315

Singapore : \$2188

Kuwait : \$2105

Australia : \$1329

Brunei : \$1272
\end{enumerate}

\textbf{Extra Credit:}
\begin{enumerate}
\item Top 10 countries with highest ratio of population to number of airports:

Bangladesh,
Hong Kong,
India,
Nigeria,
China,
Ghana,
Vietnam,
Gambia
Rwanda,
Ethiopia

\item 
If (0,0) is defined as the coordinate 90 N, 180 W, 
 then the box is between the coordinates 
(114, 70)
(124, 80)

Countries within this box:

Anguilla,
Antigua and Barbuda,
Barbados,
British Virgin Islands,
Dominica,
Grenada,
Montserrat,
Puerto Rico,
Saint Barthelemy,
Saint Kitts and Nevis,
Saint Lucia,
Saint Vincent and the Grenadines,
Sint Maarten,
Trinidad and Tobago,
Venezuela,
Virgin Islands

\end{enumerate}
\end{document}